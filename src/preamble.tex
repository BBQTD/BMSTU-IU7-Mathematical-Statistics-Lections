%%% Работа с русским языком
\usepackage{cmap}					% поиск в PDF
\usepackage{mathtext} 				% русские буквы в формулах
\usepackage[T2A, TS1]{fontenc}			% кодировка
\usepackage[utf8]{inputenc}			% кодировка исходного текста
\usepackage[english, russian]{babel}	% локализация и переносы
\usepackage{color} 				 	% цветные буковки



\usepackage{hyperref}



%%% Математика
\usepackage{amsmath, amsfonts, amssymb, mathtools} % AMS
\usepackage{cancel} % \cancel зачеркивание в формулах
\usepackage{xfrac}
\usepackage{array}



\newcommand{\vvec}[1]{%
	\ensuremath{\overrightarrow{#1}}}



%%% Теоремы, Определения
\usepackage{amsthm}

\theoremstyle{definition} % "Определение"
\newtheorem{thm}{Теорема}[section]
\newtheorem{defn}{Определение}[section]
\newtheorem{exm}{Пример}[section]
\newtheorem*{rem}{Замечание}
\newtheorem*{slv}{Решение}
\newtheorem*{cor}{Следствие}



%%% Форматирование
% Выделение + курсив к куску текста
\newcommand{\bi}[1]{%
	\textbf{\textit{#1}}}

% Математическое ожидание
\newcommand{\m}[1]{%
	\ensuremath{M\!#1}}

% Математическое ожидание X
\newcommand{\mx}{%
	\ensuremath{M\!X}}

\newcommand{\mxx}{%
	\ensuremath{M\!X^2}}

% Математическое ожидание Y
\newcommand{\my}{%
	\ensuremath{M\!Y}}

\newcommand{\myy}{%
	\ensuremath{M\!Y^2}}

% Дисперсия
\newcommand{\disp}[1]{%
	\ensuremath{D\!#1}}

% Дисперсия X
\newcommand{\dx}{%
	\ensuremath{D\!X}}

% Дисперсия Y
\newcommand{\dy}{%
	\ensuremath{D\!Y}}

% Следовательно
\newcommand{\Rarrow}{%
	\ensuremath{\;\Rightarrow\;}}

% Бесконечная последовательность 
\newcommand{\infseq}[3]{%
	\ensuremath{#1_#2, \dots, #1_#3, \dots}\ }

% Бесконечная последовательность X_1, ... X_n, ...
\newcommand{\infseqX}{%
	\infseq{X}{1}{n}}


\usepackage{tikz}



%%% Поля страницы
\usepackage[top=20mm, bottom=20mm, left=30mm, right=15mm]{geometry}



%%% Вставка иллюстраций
\usepackage{graphicx}
\graphicspath{{img/}}