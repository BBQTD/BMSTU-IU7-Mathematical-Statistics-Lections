% !TEX root = ../main.tex

\begin{center}
	\section*{Математическая Статистика}
\end{center}



\section{Основные понятия выборочной теории}


\subsection{Основные определения}

Теория вероятностей является одной из областей ``чистой'' математики, которая строится дедуктивно, исходя из вполне определённых аксиом.

Математическая статистика является разделом прикладной математики, которая строится индуктивно: от наблюдения к гипотезе, при этом аргументация основана на выводах теории вероятностей.

\paragraph{Типовая задача теории вероятностей}
При одном подбрасывании монеты вероятность выпадения герба равна $p$. Какова вероятность того, что при $n$ подбрасываниях герб выпадет $m$ раз?

\paragraph{Типовая задача математической статистики}
При $n$ подбрасываниях монеты герб выпал $m$ раз. Чему равна вероятность $p$ выпадания герба при одном подбрасывании?

\paragraph{Основная задача математической статистики} Разработка методов получения обоснованных выводов о массовых явлениях и процессах по результатом наблюдений или экспериментов. Эти выводы относятся не к результатам отдельных экспериментов, а предоставляют собой вероятностные характеристики случайных явлений.

\paragraph{``Общая'' задача математической статистики} $X$ является случайной величиной, законы распределения которой не известны. Требуется по данным наблюдений (или экспериментов) за реализациями \emph{случайной величины} $X$ сделать выводы о её законе распределения

\begin{defn}
	Множество возможных значений случайной величины $X$ называется \emph{генеральной совокупностью}
\end{defn}

\begin{defn}
	\emph{Случайной выборкой} из генеральной совокупности $X$ называется \emph{вектор}
	\begin{equation}
		\vvec{X_n} = (X_1, \dots, X_n), \quad \text{где}
	\end{equation}
	\begin{itemize}
		\item $X_i$, $i \in \aleph$ --- независимые (в совокупности) случайные величины, имеющие то же распределение, что и генеральная совокупность $X$.
	\end{itemize}
\end{defn}
\begin{rem}
	При этом $n$ называется \emph{объёмом случайной выборки \vvec{X_n}}.
\end{rem}
\begin{rem}
	Пусть $F(t)$ --- \emph{функция распределения случайной величины $X$}. Тогда \emph{функция распределения случайной выборки $\vvec{X_n}$} имеет вид
	\begin{multline}
		F_{\vvec{X_n}(x_1, \dots, x_n)} = \Prob \{ X_1 < x_1, \dots, X_n < x_n \} = \\
		= \Prob \{ X_1 < x_1 \} \cdot \ldots \cdot \Prob \{ X_n < x_n \} = F(x_1) \cdot \ldots \cdot F(x_n)
	\end{multline} 
\end{rem}

\begin{defn}
	Любую возможную реализацию $\vec{x}_n = (x_1, \dots, x_n)$ случайной выборки $\vvec{X_n}$ называют \emph{выборкой объёма $n$} для \emph{случайной величины $X$}.
\end{defn}
\begin{rem}
	При этом $x_k$ --- $k$-ый элемент выборки $\vec{x}_n$.
\end{rem}

\begin{defn}
		Множество всех возможных значений случайного вектора $\vvec{X_n}$ называют \emph{выборочным пространством}
\end{defn}

\begin{defn}
	Любую числовую  функцию $g(\vvec{X_n})$ будем называть \emph{статистикой} (или \emph{выборочной характеристикой}).
\end{defn}
\begin{rem}
	Значение $g(\vec{x}_n)$ статистики $g(\vvec{X_n})$ называют \emph{выборочным значением статистики} $g$.
\end{rem}
\begin{rem}
	Пусть $\vec{x}_n$ --- реализация случайной выборки $\vvec{X_n}$. Это позволяет моделировать случайную величину $X$ (закон распределения которой не известен) дискретной случайной величиной, ряд распределения которой имеет вид
	\begin{center}\begin{tabular}{| c || c | c | c | c | c |}
		\hline
		Значение & $x_1$ & \ldots & $x_i$ & \ldots & $x_n$ \\
		\hline
		Вероятность & $\sfrac{1}{n}$ & \ldots & $\sfrac{1}{n}$ & \ldots & $\sfrac{1}{n}$ \\
		\hline
	\end{tabular}\end{center}
	\emph{Математическое ожидание} такой случайной величины
	\[
		m = \sum_{i = 1}^{n} \frac{1}{n} \cdot x_i = \frac{1}{n} \sum_{i = 1}^{n} x_i
	\]
	\emph{Дисперсия}
	\[
		\sigma^2 = \sum_{i = 1}^{n} (x_i - m)^2 \cdot \frac{1}{n} = \frac{1}{n} \sum_{i = 1}^{n} \left(x_i - \frac{1}{n} \sum_{i = 1}^{n} x_i\right)^2
	\]
	Эти соображения приводят к следующему определениям.
\end{rem}

\begin{defn}
	\emph{Выборочным средним (выборочным математическим ожиданием)} называют статистику 
	\begin{equation}
		\dot{\mu}_1(\vvec{X_n}) = \overline{X_n} = \frac{1}{n} \sum_{i = 1}^{n} X_i
	\end{equation}
\end{defn}

\begin{defn}
	\emph{Выборочной дисперсией} называют статистику 
	\begin{equation}
		\hat{\sigma}^2(\vvec{X_n}) = \frac{1}{n} \sum_{i = 1}^{n} (X_i - \overline{X}_n)^2
	\end{equation}
\end{defn}

\begin{defn}
	\emph{Выборочным моментом $j$-го порядка} называется статистика
	\begin{equation}
		\hat{\mu}_j(\vvec{X_n}) = \frac{1}{n} \sum_{i = 1}^{n} X_i^j
	\end{equation}
\end{defn}

\begin{defn}
	\emph{выборочным центральным моментом $j$-го порядка} называется статистика
	\[
		\hat{\nu}_j(\vvec{X_n}) = \frac{1}{n} \sum_{i = 1}^{n} (X_i - \overline{X})^j
	\]
\end{defn}
\begin{rem}
	\begin{equation}
		\hat{\sigma}^2 = \hat{\nu}_2
	\end{equation}
\end{rem}


\subsection{Предварительная обработка результатов экспериментов}

\subsubsection{Вариационный ряд}

Рассмотрим $\vec{x}_n = (x_1, \dots, x_n)$ --- реализация случайной выборки. Упорядочим значения $x_1, \dots, x_n$, расположив их в порядке не убывания
\[
	x_{(1)} \leq x_{(2)} \leq \dots \leq x_{(n)}; \quad x_{(1)} = \min \{ x_1, \dots, x_n \}; \quad x_{(n)} = \max \{ x_1, \dots, x_n \}.
\]

\begin{defn}
	Последовательность $x_{(1)}, \dots, x_{(n)}$, которая удовлетворяет условию 
	\[
		x_{(1)} \leq x_{(2)} \leq \dots \leq x_{(n)}
	\] 
	называется \emph{вариационным рядом выборки $\vec{x}_n$}.
\end{defn}
\begin{rem}
	$x_{(i)}$ --- $i$-ый член вариационного ряда.
\end{rem}

\begin{defn}
	\emph{Вариационным рядом случайной выборки $\vvec{X_n}$} называется последовательность случайных величин $X_{1}, \dots, X_{n}$, где случайная величина $X_{(i)}$ для каждой реализации $\vec{x}_n$ случайной выборки $\vvec{X_n}$ принимает значение, равное значению $x_{(i)}$.
\end{defn}
\begin{rem}
	$\Prob \{ X_{(i)} \leq X_{(i+1)} \} = 1$
\end{rem}
\begin{rem}
	Пусть $F(x)$ --- функция распределения случайной величины $X$. Тогда
	\[
		F_{X_n}(x) = \Prob \{ X_{(n)} < x \} = \Prob \bigl\{ \{ X_1 < x \} \cdot \ldots \cdot \{ X_n < x \} \bigr\} =
	\]
	так как события независимы
	\[
		= \Prob \underbrace{\{ X_1 < x \}}_{F_{X_1}(x) = F(x)} \cdot \ldots \cdot \Prob \{ X_n < x \} = \bigl[ F(x) \bigr]^n
	\]
	\begin{multline*}
		F_{X_{(1)}}(x) = \Prob \{ X_{(1)} < x \} = 1 - \Prob \{ X_{(1)} \geq x \} = 1 - \Prob \{ X_1 \geq x, \ldots, X_n \geq x \} = \\
		= 1 - \Prob \{ X_1 \geq x \} \cdot \ldots \cdot \Prob \{ X_n \geq x \} = 1 - \bigl(1 - \Prob \{ X_1 \leq x \}\bigr) \cdot \ldots \cdot \bigl(1 - \Prob \{ X_n \leq x \}\bigr) = \\
		= 1 - \bigl(1 - F(x)\bigr)^n
	\end{multline*}
\end{rem}


\subsubsection{Статический ряд}

Среди элементов выборки $\vec{x}_n = (x_1, \ldots, x_n)$ могут встретиться одинаковые. Это может иметь место, например, если генеральная совокупность $X$ является дискретной случайной величиной или если $X$ непрерывная случайная величина, но при измерениях имело место округление.

Предположим, что среди значений выборки $\vec{x}_n = (x_1, \ldots, x_n)$ выделены $m$ попарно различных значений. 
\[
	z_{1} < z_{2} < \ldots < z_{m}, \quad \text{так что} \quad \forall i \in \{ 1, \ldots, n \}, \quad \exists j \in \{ 1, \ldots, m \}, \quad x_i = z_{(j)}
\]

Пусть среди компонент вектора $x_n$ ровно $n_j$ компонент приняли значение $z_j$, $j = \overline{1, m}$. Тогда таблицу
\begin{center} \begin{tabular}{| c | c | c | c | c |}
		\hline
		$z_{(1)}$ & \ldots & $z_{(j)}$ & \ldots & $z_{(m)}$ \\
		\hline
		$n_1$ & \ldots & $n_j$ & \ldots & $n_m$ \\
		\hline
\end{tabular} ,
\qquad $\sum_{j = 1}^{m} n_j = n$ \end{center}
называют \emph{статическим рядом для выборки $\vec{x}_n$}. При этом $n_j$ называют \emph{частотой значения $z_{(j)}$}, а величина $\frac{n_j}{n}$ --- \emph{относительной частотой значения $z_{(j)}$}.


\subsubsection{Эмпирическая функция распределения}

\begin{itemize}
	\item $\vvec{X_n} = (X_1, \ldots, X_n)$ --- случайная выборка
	\item $\vec{x}_n = (x_1, \dots, x_n)$ --- реализация случайной выборки $\vvec{X_n}$ 
	\item $(X_1, \ldots, X_n)$ --- независимые случайные величины;
	\item $n(x, \vec{x}_n)$ --- количество элементов выборки $\vec{x}_n$, которые меньше $x$
\end{itemize}

\begin{defn}
	\emph{Эмпирической функцией распределения} называют функцию
	\[
		F_n\colon \Re \to \Re, \quad F_n(x) = \frac{n(x, \vec{x}_n)}{n}.
	\]
\end{defn}
\begin{rem}
	$F_n(x)$ обдалает всеми свойствами функции распределения. При этом она кусочно-постоянна и принимает значения $0, \frac{1}{n}, \frac{2}{n}, \ldots, \frac{(n-1)}{n}, 1$
\end{rem}
\begin{rem}
	Если все элементы вектора $\vec{x}_n$ различны, то
	\[
		F_n(x) = 
		\begin{cases}
			0, & x \leq x_{(1)}; \\
			\frac{i}{n}, & x_{(i)} < x \leq x_{(i+1)},\; i = \overline{1, n-1}; \\
			1, & x > x_{(n)}.
		\end{cases}
	\]
\end{rem}
\begin{rem}
	Эмпирическая функция распределения позволяет интерпретировать выборку $\vec{x}_n$ как реализацию дискретной случайной величины $\widetilde{X}$ ряд распределения которой
	\begin{center}
		\renewcommand{\arraystretch}{1.5}
		\begin{tabular}{| c || c | c | c |}
		\hline
		$\widetilde{X}$ & $x_{(1)}$ & \ldots & $x_{(n)}$ \\
		\hline
		$\Prob$ & $\sfrac{1}{n}$ & \ldots & $\sfrac{1}{n}$ \\
		\hline
		\end{tabular}
	\end{center}
	В дальнейшем это позволит рассматривать числовые характеристики случайной величины $\widetilde{X}$ как приближённые значения числовых характеристик случайной величины $X$.
\end{rem}


\subsubsection{Выборочная функция распределения}

\begin{itemize}
	\item $n(x, \vvec{X_n})$ --- функция, которая для каждой реализации $\vec{x}_n$ случайной выборки $\vvec{X_n}$ принимает значение, равное числу элементов $\vec{x}_n$ которые меньше $x$
\end{itemize}

\begin{defn}
	\emph{Выборочной функцией распределения} называют функцию
	\begin{equation}
		\widehat{F}(x, \vvec{X_n}) = \frac{n(x, \vvec{X_n})}{n}
	\end{equation}
\end{defn}
\begin{rem}
	Зафиксируем некоторе значение $x \in \Re$. Тогда для этого $x$, $\widehat{F}(x, \vvec{X_n})$ является \emph{функцией случайной выборки $\vvec{X_n}$} $\Rightarrow$ является случайной величиной.
\end{rem}
\begin{rem}
	При каждом фиксированном $x \in \Re$ случайная величина $\widehat{F}(x, \vvec{X_n})$ может принимать значения
	\[
		0, \frac{1}{n}, \frac{2}{n}, \dots, \frac{n-1}{n}, 1.
	\]
\end{rem}

\noindent
Обозначим $p = \Prob \{ X_i < x \}$ --- не зависит от $i$, так как все $X_i$ одинаково распределены.

\[
	\Prob \left\{ \widehat{F}(x, \vvec{X_n}) = \frac{k}{n} \right\} = \Prob \left\{ \frac{n(x, \vvec{X_n})}{n} = \frac{k}{n} \right\} = \Prob \bigl\{ n(x, \vvec{X_n}) = k \bigr\} = 
\]

\noindent
В векторе $\vvec{X_n}$ ровно $k$ компонент принимающих значение $< x$, при этом $X_1, \dots, X_n$ --- независимы, $\Prob \{ X_i , x \}$ --- $p$ 

\[
	= C_n^k p^k (1 - p)^{n-k}
\]

\noindent
Схема Бернулли, $\;\Rightarrow\; \hat{F}(x, \vvec{X_n}) \sim B(n, p)$ --- биноминальная случайная величина при фиксированном $x$.

\begin{thm}
	$\forall x \in \Re$ последовательность $\widehat{F}(x, \vvec{X_n})$ сходится по вероятности к значению $F_{X}(x)$ bla bla bla то есть
	\begin{equation}
		\widehat{F}(x, \vvec{X_n}) \xrightarrow[n\to\infty]{\Prob} F_{X}(x)
	\end{equation}
\end{thm}
\begin{proof}
	При каждом фиксированном $x \in \Re$ величина $\widehat{F}(x, \vvec{X_n})$ равна относительной (наблюдённой) частоте реализации реализации успеха в серии из $n$ испытаний по \emph{схеме Бернулли} (успех осуществления событий $\{ X < x \}$). В соответствии с \emph{законом больших чисел в форме Бернулли}
	\[
		\widehat{F}(x, \vvec{X_n}) \xrightarrow[n\to\infty]{\Prob} F_{X}(x)
	\]
\end{proof}


\subsubsection{Интервальный статический ряд}

Выше было введено понятие \emph{статического ряда}. Однако если число наблюдений велико ($n > 50$), то их группируют не только в виде статического ряда, но и в виде интервального статического ряда. Для этого отрезок $J = [x_{(1)}, x_{(n)}]$ разбивают на $m$ равновеликих интервалов

\begin{align*}
	&J_i = \left[ x_{(1)} + (i - 1)\Delta, x_{(i)} + i\Delta \right), \quad i = \overline{1, m-1};
	\\
	&J_m = \left[ x_{(1)} + (m-1)\Delta, x_{(1)} + m\Delta  \right];
	\\
	&\Delta = \frac{|J|}{m}.
\end{align*}

\begin{rem}
	При выборе $m$ обычно используют формулу
	\begin{equation}
		m = [ \log_2 n ] + 1, \quad \text{где}
	\end{equation}
	\begin{itemize}
		\item $[\alpha]$ --- целая часть от $\alpha$
	\end{itemize}
\end{rem}

\begin{defn}
	\emph{Интервальным статическим рядом} называют таблицу
	\begin{center}
			\begin{tabular}{| c | c | c |}
				\hline
				$J_1$ & \ldots & $J_m$ \\
				\hline
				$n_1$ & \ldots & $n_m$ \\
				\hline
			\end{tabular}
	\end{center}
	\begin{itemize}
		\item $n_1 + \dots + n_m = n$,
		\item $n_i$ --- количество элементов выборки $\vec{x}_n$, принадлежащих множеству $J_i$, $i = \overline{1, m}$
	\end{itemize}
\end{defn}


\subsubsection{Эмпирическая плотность}

Пусть для данной выборки $\vec{x}_n$ построим интервальный статический ряд

\begin{defn}
	\emph{Эмпирической плотностью распределения случайной величины $X$} называют функцию
	\begin{equation}
		f_n(x) =
		\begin{cases}
			\frac{n_i}{n \Delta}, &x \in J_i;\\
			0, &\text{иначе}.
		\end{cases}
	\end{equation}
\end{defn}

\begin{defn}
	График функции $f_n(x)$ называют гистограммой. (гистограмма)
\end{defn}

\begin{rem}
	\[
		\int_{-\infty}^{+\infty} f_n(x)\, dx = \sum_{i =1}^{n} \text{(площадь прямоугольников)} = \sum_{i = 1}^{n} \frac{n_i}{n \Delta} \Delta = \frac{1}{n} \sum_{i = 1}^{n} n_i = \frac{n}{n} = 1
	\]
\end{rem}
\begin{rem}
	Функция $f_n(x)$ является статическим аналогом плотности распределения случайной величины можно показать, что для непрерывной случайной величины $X$
	\[
		\forall x \in \Re, \quad f_n(x) \xrightarrow[n\to\infty]{\Prob} f_{X}(x)
	\]
	то есть при больших $n \;\Rightarrow\; f_n(x) \approx f(x)$
\end{rem}


\subsubsection{Полигональная частота}

Пусть для данной выборки $\vec{x}_n$ построена гистограмма (рисунок)

\begin{defn}
	\emph{Полигоном частот} называют ломанную звенья которой соединяют середины верхних сторон прямоугольников гистограммы. (рисунок)
\end{defn}