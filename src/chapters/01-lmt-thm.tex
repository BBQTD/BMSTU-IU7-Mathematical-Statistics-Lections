% !TEX root = ../main.tex

\section{Предельные теоремы теории вероятностей}


\subsection{Неравенства Чебышева}

\begin{thm}[\emph{первое неравенство господина Чебышева}]
	\begin{equation}
		\left. \begin{aligned}
			&X \geq 0;\\
			&\exists \mx.
		\end{aligned} \right\} \Rarrow \forall \varepsilon > 0,\; \Prob \{ X \geq \varepsilon \} \leq \frac{\mx}{\varepsilon}
	\end{equation}
	\begin{itemize}
		\item $X$ --- случайная величина;
		\item $\Prob \{ X \leq 0 \} = 0$ так как $X \geq 0$.
	\end{itemize}
\end{thm}

\begin{proof}
	Для непрерывной случайное величины $X$
	\[
		\mx = \int_{-\infty}^{+\infty} x\, f(x)\, dx =
	\]
	так как $X \geq 0$, $f(x) = 0$, $x < 0$
	\[
		= \int_{0}^{+\infty} x\, f(x)\, dx = \underbrace{\int_{0}^{\varepsilon} x\, f(x)\, dx}_{\geq 0} + \int_{\varepsilon}^{+\infty} x\, f(x)\, dx \geq \int_{\varepsilon}^{+\infty} a\, f(x)\, dx \geq
	\]
	так как $a \geq \varepsilon$
	\[
		\geq ... = \varepsilon \cdot \Prob \{ X \geq \varepsilon \}
	\]
	таким образом
	\[
		\mx \geq \varepsilon \cdot \Prob \{ X \geq \varepsilon \} \;\Rightarrow\; \Prob \{ X \geq \varepsilon \} \geq \frac{\mx}{\varepsilon}
	\]
\end{proof}

\begin{thm}[\emph{второе неравенство лорда Чебышева}]
	\begin{equation}
		\exists \mx, \exists \dx \Rarrow \forall \varepsilon > 0,\; \Prob \left\{ |X - \mx| \geq \varepsilon \right\} \leq \frac{\dx}{\varepsilon^2}
	\end{equation}
	\begin{itemize}
		\item $X$ --- случайная величина.
	\end{itemize}
\end{thm}