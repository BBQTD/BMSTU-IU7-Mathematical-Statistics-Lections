% !TEX root = ../main.tex

\section{Предельные теоремы теории вероятностей}


\subsection{Неравенства Чебышева}

\begin{thm}[\emph{первое неравенство господина Чебышева}]
	\begin{equation}
		\left. \begin{aligned}
			&X \geq 0;\\
			&\exists \mx.
		\end{aligned} \right\} \Rarrow \forall \varepsilon > 0,\; \Prob \{ X \geq \varepsilon \} \leq \frac{\mx}{\varepsilon}
	\end{equation}
	\begin{itemize}
		\item $X$ --- случайная величина;
		\item $\Prob \{ X \leq 0 \} = 0$ так как $X \geq 0$.
	\end{itemize}
\end{thm}

\begin{proof}
	Для непрерывной случайное величины $X$ и зная, что при $X \geq 0 \Rarrow f(x) = 0$, $x < 0$
	\[
		\mx = \int_{-\infty}^{+\infty} x\, f(x)\, dx = \int_{0}^{+\infty} x\, f(x)\, dx = \underbrace{\int_{0}^{\varepsilon} x\, f(x)\, dx}_{\geq 0} + \int_{\varepsilon}^{+\infty} x\, f(x)\, dx
	\]
	учитывая $x \geq \varepsilon$
	\[
		\underbrace{\int_{0}^{\varepsilon} x\, f(x)\, dx}_{\geq 0} + \int_{\varepsilon}^{+\infty} x\, f(x)\, dx \geq \int_{\varepsilon}^{+\infty} x\, f(x)\, dx \geq \varepsilon\cdot\int_{\varepsilon}^{+\infty} f(x)\, dx
	\]
	где
	\[
		\varepsilon\cdot\int_{\varepsilon}^{+\infty} f(x)\, dx = \varepsilon\cdot\Prob\{ X \geq \varepsilon \}
	\]
	таким образом
	\[
		\mx \geq \varepsilon \cdot \Prob \{ X \geq \varepsilon \} \;\Rightarrow\; \Prob \{ X \geq \varepsilon \} \leq \frac{\mx}{\varepsilon}
	\]
\end{proof}

\begin{thm}[\emph{второе неравенство лорда Чебышева}]
	\begin{equation}
		\exists \mx, \exists \dx \Rarrow \forall \varepsilon > 0,\; \Prob \bigl\{ |X - \mx| \geq \varepsilon \bigr\} \leq \frac{\dx}{\varepsilon^2}
	\end{equation}
	\begin{itemize}
		\item $X$ --- случайная величина.
	\end{itemize}
\end{thm}

\begin{proof} Выпишем дисперсию
	\[
		\dx = M\bigl[ (X - \mx)^2 \bigr]
	\]
	Рассмотрим случайную величину $Y = (X - \mx)^2$, где $Y \geq 0$. Тогда из \emph{первого неравенства Чебышева} следует, что $\forall\delta\geq 0$, $\my\geq\delta\Prob\{ Y\geq\delta \}$, где получается, что $\delta = \varepsilon^2$.
	\[
		\Bigl[\dx = M\bigl[ (X - \mx)^2 \bigr]\Bigr] \geq \Bigl[\varepsilon^2\cdot\Prob\bigl\{ (X - \mx)^2\geq\varepsilon^2 \bigr\} = \varepsilon^2\cdot\Prob\bigl\{ |X - \mx|\geq\varepsilon \bigr\}\Bigr]
	\]
	таким образом
	\[
		\dx \geq \varepsilon^2\cdot\Prob\bigl\{ |X - \mx|\geq\varepsilon \bigr\} \Rarrow \Prob\bigl\{ |X - \mx|\geq\varepsilon \bigr\} \leq \frac{\dx}{\varepsilon^2}
	\]
\end{proof}

\begin{exm}
	Предельно допустимое давление в пневмосистеме ракеты равна $200$ (Па). После проверки большого количество ракет было получено среднее значение давления 150 (Па). Оценить вероятность того, что давление в пневмосистеме очередной ракеты будет больше $200$ (Па), если по результатам проверки  ракет было получено среднеквадратичное отклонение 5 (Па).
\end{exm}

\begin{slv}
	Имеем следующее:
	\begin{itemize}
		\item случайная величина $X$ --- давление в пневмосистеме;
		\item $X \geq 0$;
		\item $\mx = 150$ (Па);
		\item $\dx = 25$ (Па);
	\end{itemize}
	Решим поставленную задачу с помощью \emph{первого неравенства Чебышева}
	\begin{align*}
		\biggl[\Prob\{ X \geq \varepsilon \} = \Prob\{ X \geq 200 \}\biggr] &\leq \biggl[\frac{\mx}{\varepsilon} = \frac{150}{200} = \frac{3}{4} = 0.75\biggr] 
		\\
		\Prob\{ X \geq 200 \} &\leq 0.75
	\end{align*}
	Поскольку нам известна дисперсия почему бы не воспользоваться \emph{вторым неравенством Чебышева}? Действуем. Для начало рассмотрим вероятность следующего события 
	\[
		\Prob\{ X \geq \varepsilon \} = \Prob\{ X \geq 200 \} = \Prob\{ X - \underbrace{150}_{\mx} \geq \underbrace{50}_\varepsilon \}
	\]
	Остаётся построить вероятность, которая будет удовлетворять форме \emph{второго неравенства Чебышева} (т.\,е. сделать модуль).
	\[
		 \Prob\{ X - 150 \geq 50 \} \leq \Prob\{ X - 150 \geq 50 \} + \Prob\{ X - 150 \leq -50 \}
	\]
	Так как \emph{события} $\{ X - 150 \geq 50 \}$ и $\{ X - 150 \leq -50 \}$ \emph{несовместные}, то по \emph{формуле сложения вероятностей несовместных событий} получаем
	\begin{multline*}
		\Prob\{ X - 150 \geq 50 \} + \Prob\{ X - 150 \leq -50 \} = \\
		= \Prob\bigl\{ \{ X - 150 \geq 50 \} + \{ X - 150 \leq -50 \} \bigr\} = \Prob\bigl\{ |X - 150| \geq 50 \bigr\}
	\end{multline*}
	Таким образом применяем \emph{второе неравенство Чебышева}
	\begin{align*}
		\biggl[\Prob\bigl\{ |X - \mx|\geq\varepsilon \bigr\} = \Prob\bigl\{ |X - 150| \geq 50 \bigr\}\biggr] &\leq \biggl[\frac{\dx}{\varepsilon^2} = \frac{25}{50^2} = \biggl(\frac{5}{50}\biggr)^2 = 0.01\biggr]
		\\
		\Prob\bigl\{ |X - 150| \geq 50 \bigr\} &\leq 0.01
	\end{align*}
	\textbf{Ответ:} \begin{itemize}
		\item с использованием \emph{первого неравенства Чебышева} $\Prob \leq 0.75$;
		\item с использованием \emph{второго неравенства Чебышева} $\Prob \leq 0.01$.
	\end{itemize}
\end{slv}

\begin{rem}
	\emph{Второе неравенство Чебышева} даёт более точную оценку, так как используется информация о дисперсии случайной величины.
\end{rem}

\begin{rem}
	Использование \emph{первого неравенства Чебышева} при $\varepsilon < \mx$ и \emph{второго неравенства Чебышева} при $\varepsilon < \sqrt{\dx}$ даёт тривиальную оценку: $\Prob \leq 1$.
\end{rem}