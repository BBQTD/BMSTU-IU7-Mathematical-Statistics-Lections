% !TEX root = ../main.tex

\section{Предельные теоремы теории вероятностей}


\subsection{Неравенства Чебышева}

\begin{thm}[\emph{первое неравенство господина Чебышева}]
	\begin{equation}
		\left. \begin{aligned}
			&X \geq 0;\\
			&\exists \mx.
		\end{aligned} \right\} \Rarrow \forall \varepsilon > 0,\; \Prob \{ X \geq \varepsilon \} \leq \frac{\mx}{\varepsilon}
	\end{equation}
	\begin{itemize}
		\item $X$ --- случайная величина;
		\item $\Prob \{ X \leq 0 \} = 0$ так как $X \geq 0$.
	\end{itemize}
\end{thm}

\begin{proof}
	Для непрерывной случайное величины $X$ и зная, что при $X \geq 0 \Rarrow f(x) = 0$, $x < 0$
	\[
		\mx = \int_{-\infty}^{+\infty} x\, f(x)\, dx = \int_{0}^{+\infty} x\, f(x)\, dx = \underbrace{\int_{0}^{\varepsilon} x\, f(x)\, dx}_{\geq 0} + \int_{\varepsilon}^{+\infty} x\, f(x)\, dx
	\]
	учитывая $x \geq \varepsilon$
	\[
		\underbrace{\int_{0}^{\varepsilon} x\, f(x)\, dx}_{\geq 0} + \int_{\varepsilon}^{+\infty} x\, f(x)\, dx \geq \int_{\varepsilon}^{+\infty} x\, f(x)\, dx \geq \varepsilon\cdot\int_{\varepsilon}^{+\infty} f(x)\, dx
	\]
	где
	\[
		\varepsilon\cdot\int_{\varepsilon}^{+\infty} f(x)\, dx = \varepsilon\cdot\Prob\{ X \geq \varepsilon \}
	\]
	таким образом
	\[
		\mx \geq \varepsilon \cdot \Prob \{ X \geq \varepsilon \} \;\Rightarrow\; \Prob \{ X \geq \varepsilon \} \leq \frac{\mx}{\varepsilon}
	\]
\end{proof}

\begin{thm}[\emph{второе неравенство лорда Чебышева}]
	\begin{equation}
		\exists \mx, \exists \dx \Rarrow \forall \varepsilon > 0,\; \Prob \bigl\{ |X - \mx| \geq \varepsilon \bigr\} \leq \frac{\dx}{\varepsilon^2}
	\end{equation}
	\begin{itemize}
		\item $X$ --- случайная величина.
	\end{itemize}
\end{thm}

\begin{proof} Выпишем дисперсию
	\[
		\dx = M\bigl[ (X - \mx)^2 \bigr]
	\]
	Рассмотрим случайную величину $Y = (X - \mx)^2$, где $Y \geq 0$. Тогда из \emph{первого неравенства Чебышева} следует, что $\forall\delta\geq 0$, $\my\geq\delta\Prob\{ Y\geq\delta \}$, где получается, что $\delta = \varepsilon^2$.
	\[
		\Bigl[\dx = M\bigl[ (X - \mx)^2 \bigr]\Bigr] \geq \Bigl[\varepsilon^2\cdot\Prob\bigl\{ (X - \mx)^2\geq\varepsilon^2 \bigr\} = \varepsilon^2\cdot\Prob\bigl\{ |X - \mx|\geq\varepsilon \bigr\}\Bigr]
	\]
	таким образом
	\[
		\dx \geq \varepsilon^2\cdot\Prob\bigl\{ |X - \mx|\geq\varepsilon \bigr\} \Rarrow \Prob\bigl\{ |X - \mx|\geq\varepsilon \bigr\} \leq \frac{\dx}{\varepsilon^2}
	\]
\end{proof}

\begin{exm}
	Предельно допустимое давление в пневмосистеме ракеты равна $200$ (Па). После проверки большого количество ракет было получено среднее значение давления 150 (Па). Оценить вероятность того, что давление в пневмосистеме очередной ракеты будет больше $200$ (Па), если по результатам проверки  ракет было получено среднеквадратичное отклонение 5 (Па).
\end{exm}

\begin{slv}
	Имеем следующее:
	\begin{itemize}
		\item случайная величина $X$ --- давление в пневмосистеме;
		\item $X \geq 0$;
		\item $\mx = 150$ (Па);
		\item $\dx = 25$ (Па);
	\end{itemize}
	Решим поставленную задачу с помощью \emph{первого неравенства Чебышева}
	\begin{align*}
		\biggl[\Prob\{ X \geq \varepsilon \} = \Prob\{ X \geq 200 \}\biggr] &\leq \biggl[\frac{\mx}{\varepsilon} = \frac{150}{200} = \frac{3}{4} = 0.75\biggr] 
		\\
		\Prob\{ X \geq 200 \} &\leq 0.75
	\end{align*}
	Поскольку нам известна дисперсия почему бы не воспользоваться \emph{вторым неравенством Чебышева}? Действуем. Для начало рассмотрим вероятность следующего события 
	\[
		\Prob\{ X \geq \varepsilon \} = \Prob\{ X \geq 200 \} = \Prob\{ X - \underbrace{150}_{\mx} \geq \underbrace{50}_\varepsilon \}
	\]
	Остаётся построить вероятность, которая будет удовлетворять форме \emph{второго неравенства Чебышева} (т.\,е. сделать модуль).
	\[
		 \Prob\{ X - 150 \geq 50 \} \leq \Prob\{ X - 150 \geq 50 \} + \Prob\{ X - 150 \leq -50 \}
	\]
	Так как \emph{события} $\{ X - 150 \geq 50 \}$ и $\{ X - 150 \leq -50 \}$ \emph{несовместные}, то по \emph{формуле сложения вероятностей несовместных событий} получаем
	\begin{multline*}
		\Prob\{ X - 150 \geq 50 \} + \Prob\{ X - 150 \leq -50 \} = \\
		= \Prob\bigl\{ \{ X - 150 \geq 50 \} + \{ X - 150 \leq -50 \} \bigr\} = \Prob\bigl\{ |X - 150| \geq 50 \bigr\}
	\end{multline*}
	Таким образом применяем \emph{второе неравенство Чебышева}
	\begin{align*}
		\biggl[\Prob\bigl\{ |X - \mx|\geq\varepsilon \bigr\} = \Prob\bigl\{ |X - 150| \geq 50 \bigr\}\biggr] &\leq \biggl[\frac{\dx}{\varepsilon^2} = \frac{25}{50^2} = \biggl(\frac{5}{50}\biggr)^2 = 0.01\biggr]
		\\
		\Prob\bigl\{ |X - 150| \geq 50 \bigr\} &\leq 0.01
	\end{align*}
	\textbf{Ответ:} \begin{itemize}
		\item с использованием \emph{первого неравенства Чебышева} $\Prob \leq 0.75$;
		\item с использованием \emph{второго неравенства Чебышева} $\Prob \leq 0.01$.
	\end{itemize}
\end{slv}

\begin{rem}
	\emph{Второе неравенство Чебышева} даёт более точную оценку, так как используется информация о дисперсии случайной величины.
\end{rem}

\begin{rem}
	Использование \emph{первого неравенства Чебышева} при $\varepsilon < \mx$ и \emph{второго неравенства Чебышева} при $\varepsilon < \sqrt{\dx}$ даёт тривиальную оценку: $\Prob \leq 1$.
\end{rem}


\subsection{Сходимость последовательности случайных величин}

Будем считать, что \infseqX --- \emph{последовательность случайных величин}, заданных на одном вероятностном пространстве.

\begin{defn}
	Последовательность случайных величин \infseqX \emph{сходится по вероятности} к случайной величине $Z$, если $\forall\varepsilon>0,\;\Prob\bigl\{ |X_n - Z| \geq\varepsilon\bigr \}\xrightarrow[n\to\infty]{}0$ Обозначение:
	\[
		X_n\xrightarrow[n\to\infty]{\Prob}Z
	\]
\end{defn}

\begin{defn}
	Последовательность случайных величин \infseqX \emph{слабо сходится} к случайной величине $Z$, если $\forall x\in\Re$ где $F_z$ непрерывна в точке $x$, числовая последовательность $F_{X_1}(x), \dots, F_{X_n}(x), \dots$ сходится к $F_Z(x)$. Обозначение:
	\[
		F_{X_n}(x) \xRightarrow[n\to\infty]{} F_Z(x)
	\]
\end{defn}

\begin{rem}
	Данные виды сходимости \emph{неэквивалентны}.
\end{rem}


\subsection{Закон больших чисел (ЗБЧ)}

Будем считать, что 
\begin{itemize}
	\item \infseqX --- \emph{последовательность случайных величин};
	\item $\exists\mx_i = m_i$, где $i = \overline{1, \infty}$.
\end{itemize}

\begin{defn}
	Последовательность случайных величин \infseqX удовлетворяет \emph{закону больших чисел}, если 
	\[\forall\varepsilon > 0,\; \Prob\Biggl\{ \Biggl|\frac{1}{n}\cdot\sum_{i = 1}^{n} X_i - \frac{1}{n}\cdot\sum_{i = 1}^{n} m_i\Biggr| \geq \varepsilon \Biggr\} \xrightarrow[n\to\infty]{} 0\]
\end{defn}

\begin{rem}
	Выполнение \emph{закона больших чисел} для последовательности \infseqX означает, что при достаточно больших $n$ величина
	\[
		Y_n = \frac{1}{n}\cdot\sum_{i = 1}^{n} X_i - \frac{1}{n}\cdot\sum_{i = 1}^{n} m_i
	\]
	практически теряет случайный характер.
\end{rem}

\begin{thm}[\emph{Закон больших чисел в форме Чебышева} или \emph{достаточное условие выполнимости для последовательности случайных величин}] Последовательность случайных величин \infseqX удовлетворяет \emph{закону больших чисел} тогда и только тогда, когда выполняются следующие условия:
	\begin{itemize}
		\item случайные величины \infseqX --- независимы;
		\item $\exists\mx_i = m_i$, $\exists\dx_i = \sigma_i^2$, $i = 1, 2, \dots$;
		\item Дисперсия случайных величин ограничена в совокупности т.\,е. 
		\[
			\exists C > 0\; \colon \sigma_i^2 \leq C, \quad i = 1, 2, \dots
		\]
	\end{itemize}
\end{thm}

\begin{proof} 
	\begin{align*}
		\overline{X}_n &= \frac{1}{n}\cdot\sum_{i = 1}^{n} X_i \quad n \in N
		\\
		\m{\overline{X}_n} &= \frac{1}{n}\cdot\sum_{i = 1}^{n} m_i
		\\
		\disp{\overline{X}_n} &= \frac{1}{n^2}\cdot D\biggl(\sum_{i = 1}^{n} X_i\biggr) = \{ X_i \text{ независимы}\} = \frac{1}{n^2}\cdot\sum_{i = 1}^{n} \disp{X_i} = \frac{1}{n^2}\cdot\sum_{i = 1}^{n} \sigma_i^2
	\end{align*}
	Используем \emph{второе неравенство Чебышева}
	\[
		\Prob \bigl\{ |\overline{X}_n - \m{\overline{X}_n}| \geq \varepsilon \bigr\} \leq \frac{\disp{\overline{X}_n}}{\varepsilon^2}
	\]
	В нашем случае
	\[
		\Prob \Biggl\{ \Biggl|\frac{1}{n}\cdot\sum_{i = 1}^{n} X_i - \frac{1}{n}\cdot\sum_{i = 1}^{n} m_i\Biggr| \geq \varepsilon\Biggr\} \leq \frac{1}{\varepsilon^2}\cdot\frac{1}{n^2}\cdot\sum_{i = 1}^{n} \sigma_i^2 \leq \frac{1}{\varepsilon^2}\cdot\frac{1}{n^2}\cdot\sum_{i = 1}^{n} C = \frac{1}{\varepsilon^2}\cdot\frac{1}{n^2}\cdot n\,C = \frac{C}{\varepsilon^2\,n}
	\]
	таким образом
	\[
		0 \leq \Prob \Biggl\{ \Biggl|\frac{1}{n}\cdot\sum_{i = 1}^{n} X_i - \frac{1}{n}\cdot\sum_{i = 1}^{n} m_i\Biggr| \geq \varepsilon\Biggr\} \leq \frac{C}{\varepsilon^2\,n}
	\]
	По \emph{теореме о двух милиционерах}
	\[
		\Prob \Biggl\{ \Biggl|\frac{1}{n}\cdot\sum_{i = 1}^{n} X_i - \frac{1}{n}\cdot\sum_{i = 1}^{n} m_i\Biggr| > \varepsilon\Biggr\} \xrightarrow[n\to\infty]{} 0
	\]
\end{proof}

\begin{cor} Пусть
	\begin{itemize} 
		\item выполняется теорема о ЗБЧ в форме Чебышева
		\item $X_i$ одинаково распределены т.\,е. $\m{X_i} = m$, $\disp{X_i} = \sigma^2$, $i \in \aleph$
	\end{itemize}
	Тогда 
	\begin{equation}
		\Prob\Biggl\{ \biggl| \frac{1}{n} \cdot \sum_{i = 1}^{n} X_i - m \biggr| > \varepsilon \Biggr\} \xrightarrow[n\to\infty]{} 0
	\end{equation}
\end{cor}

\begin{cor}[Теорема Бернулли, ЗБЧ в форме Бернулли] Пусть
	\begin{itemize}
		\item проводится серия испытаний по \emph{схеме Бернулли}
		\begin{itemize}
			\item с вероятностью успеха $p$;
			\item c вероятностью неудачи $q = 1 - p$;
		\end{itemize}
		\item наблюденная частота успеха $r_n = \{ \text{число успехов в первых n испытаниях} \}/n$. 
	\end{itemize}
	Тогда
	\begin{equation}
		r_n \xrightarrow[n\to\infty]{\Prob} p
	\end{equation}
\end{cor}

\begin{proof}
	Рассмотрим случайную величину
	\[
		X_i = \begin{cases*}
		1, & \text{если в $i$-ом испытании успех;} \\
		0, & \text{иначе.}
		\end{cases*}
	\]
	тогда $\m{X_i} = p$, $\disp{X_i} = p\,q$. Поскольку $\exists\m{X_i}$, $\exists\disp{X_i}$ и $X_i$ --- независимы по определению \emph{схемы Бернулли} $\Rarrow$ выполняются все условия предыдущего следствия
	\[
		\Prob\Biggl\{ \biggl| \underbrace{\frac{1}{n} \cdot \sum_{i = 1}^{n} X_i}_{r_n} -\, p\, \biggr| > \varepsilon \Biggr\} \xrightarrow[n\to\infty]{} 0
	\]
	Таким образом
	\[
		r_n \xrightarrow[n\to\infty]{\Prob} p
	\]
\end{proof}